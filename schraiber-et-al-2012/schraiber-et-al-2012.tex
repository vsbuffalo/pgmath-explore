\documentclass[12pt]{article}\usepackage[]{graphicx}\usepackage[]{color}
%% maxwidth is the original width if it is less than linewidth
%% otherwise use linewidth (to make sure the graphics do not exceed the margin)
\makeatletter
\def\maxwidth{ %
  \ifdim\Gin@nat@width>\linewidth
    \linewidth
  \else
    \Gin@nat@width
  \fi
}
\makeatother

\definecolor{fgcolor}{rgb}{0.345, 0.345, 0.345}
\newcommand{\hlnum}[1]{\textcolor[rgb]{0.686,0.059,0.569}{#1}}%
\newcommand{\hlstr}[1]{\textcolor[rgb]{0.192,0.494,0.8}{#1}}%
\newcommand{\hlcom}[1]{\textcolor[rgb]{0.678,0.584,0.686}{\textit{#1}}}%
\newcommand{\hlopt}[1]{\textcolor[rgb]{0,0,0}{#1}}%
\newcommand{\hlstd}[1]{\textcolor[rgb]{0.345,0.345,0.345}{#1}}%
\newcommand{\hlkwa}[1]{\textcolor[rgb]{0.161,0.373,0.58}{\textbf{#1}}}%
\newcommand{\hlkwb}[1]{\textcolor[rgb]{0.69,0.353,0.396}{#1}}%
\newcommand{\hlkwc}[1]{\textcolor[rgb]{0.333,0.667,0.333}{#1}}%
\newcommand{\hlkwd}[1]{\textcolor[rgb]{0.737,0.353,0.396}{\textbf{#1}}}%

\usepackage{framed}
\makeatletter
\newenvironment{kframe}{%
 \def\at@end@of@kframe{}%
 \ifinner\ifhmode%
  \def\at@end@of@kframe{\end{minipage}}%
  \begin{minipage}{\columnwidth}%
 \fi\fi%
 \def\FrameCommand##1{\hskip\@totalleftmargin \hskip-\fboxsep
 \colorbox{shadecolor}{##1}\hskip-\fboxsep
     % There is no \\@totalrightmargin, so:
     \hskip-\linewidth \hskip-\@totalleftmargin \hskip\columnwidth}%
 \MakeFramed {\advance\hsize-\width
   \@totalleftmargin\z@ \linewidth\hsize
   \@setminipage}}%
 {\par\unskip\endMakeFramed%
 \at@end@of@kframe}
\makeatother

\definecolor{shadecolor}{rgb}{.97, .97, .97}
\definecolor{messagecolor}{rgb}{0, 0, 0}
\definecolor{warningcolor}{rgb}{1, 0, 1}
\definecolor{errorcolor}{rgb}{1, 0, 0}
\newenvironment{knitrout}{}{} % an empty environment to be redefined in TeX

\usepackage{alltt}
\usepackage[round]{natbib}
\usepackage{graphicx}
\usepackage{hyperref}
\usepackage{fullpage}
\usepackage{amsmath}

\hypersetup{
  colorlinks=true,
  pdfborder={1, 1, 1}
}
\IfFileExists{upquote.sty}{\usepackage{upquote}}{}

\begin{document}

\title{Notes on Schraiber et al., 2012 - \emph{Genomic Tests on Variation in Inbreeding Among Individuals and Among Chromosomes}}
\author{Vince Buffalo}
\maketitle
\section{Levene's 1949 Model: The Probability of Heterozygotes with No Inbreeding}

First, we consider equation (1) in \citet{Schraiber:2012it}, the
probability of seeing some number heterozygotes in a sample with no
inbreeding derived from \citep{Levene:1949va} (but I went back to
\citet{Haldane:1954ts} and saw the same derivation there,
earlier). The setup is that we have $n$ diploid individuals, giving us
$n$ possible ``bins'' to put $2n$ balls into. If we only consider
polymorphic sites (assuming biallelic), there are two alleles:
ancestral and derived. For our $2n$ alleles in our sample, call $i$
the number of derived alleles. $i$ is bound between $1$ and $2n-1$
(that is, $i \in \{1, 2, \ldots, 2n-1\}$) since if $i=0$, we have no
derived alleles and our site is not polymorphic, and if $i=2n$, our
site is entirely derived alleles, and again, our site is not
polymorphic.

There are three additional constraints. First, if we define $h$ as the
number of heterozygous individuals (that is, individuals with one of
the $i$ derived alleles and one of the $2n-i$ ancestral alleles), then
if $h$ is odd, $i$ must be odd, and second, if $h$ is even, then $i$
must be even. If this is unclear, consider the case where $h$ is
even. If an additional derived allele is added, it either goes into a
heterozygous individual, and makes it homozygous (so $h' = h - 1$,
making $h$ odd) or it makes a homozygous individual heterozygous (so
$h' = h + 1$, making $h$ odd). Finally, the last constraint is that $h
\le \min(i, 2n-i)$, which can be understood as the most heterozygotes
you can have is the minimum number of derived or ancestral
alleles. This is easy intuitively too: in the extreme case that if
\emph{every} minor allele ($i$ or $2n-i$) was in a heterozygous
individual, then the most heterozygous genotypes possible is the
number of minor alleles.

\subsection{Combinatorics}

Schraiber uses a classic colored ball and bin combinatorial approach
equation borrowed on \cite{Levene:1949va} and \cite{Haldane:1954ts} to
look at the probability of heterozygotes given our seen number of
derived alleles, $i$, and our sample size, $n$. Combinatorial
approaches are very powerful, so I will go into some depth in this
derivation.

When deriving probabilities using combinatorial approaches, our goal
is to enumerate all outcomes in the sample space to determine the
denominator, and then derive an expression for the number of outcomes
for a particular event as the numerator. So to derive the probability
of $h$ heterozygotes given we see $i$ derived alleles in a sample of
$n$, we first want to determine the denominator by asking how many
total ways are there are to arrange $2n - i$ red balls (ancestral
alleles) and $i$ white balls (derived alleles). 

One way of thinking about this is to take all arrangements of $2n$
balls, treating each ball as distinguishable (there are $(2n)!$ of
these). Since we don't care about the arrangement of indistinguishable
white balls, we divide the number of arrangements of $i$ white balls
($i!$). We also don't care about the number of arrangements of $2n -
i$ indistinguishable red balls, so we divide out $(2n - i)!$ red
balls. This leaves us with:

\begin{equation} \label{eq:01}
\frac{(2n)!}{i! (2n - i)!}
\end{equation}

arrangements of $i$ white balls and $2n - i$ red balls, the well-known
binomial coefficient. In this case, our problem can be set up this
way: given numbered (distinguishable) balls, how many ways are there
to grab $i$ and paint them white? Note that certain arrangements of
balls, i.e. for $n = 5$ individuals we will have arrangements that are
lead to the same genotype: $(0, 1) (1, 0), (0, 0) (1, 1) (0, 1)$ and
$(1, 0) (1, 0), (0, 0) (1, 1) (0, 1)$ are the same. We will account
for this in a bit.

Now, let $n_2$, $n_1$, and $n_0$ be the number of bins with 2, 1, and
0 white balls. Immediately, notice that $n_1 = h$, the number of
heterozygotes, and $n_0 + n_1 + n_2 = n$.

Since $i$ is the number of white balls (derived alleles), the number
of $n_0$ bins (ancestral homozygous, bins with only red balls) is $n_0
= (2n - i - h)/2$, or intuitively, the number of ancestral (red) balls
($2n - i$), minus the number in heterozygotes ($h$), leaving the
number of ancestral balls in homozygous bins. Since we want a count of
the number of \emph{bins}, and not \emph{balls}, and these are by
definition in homozygous state, we divide by 2.

Next, for $n_2$, the number of homozygous derived (bins with two white
balls) we use the same logic. There are $i$ white balls, but we lose
$h$ of these two homozygous bins. Thus, we are left with $(i - h)/2$
bins that are homozygous derived.

Next, $n_2$, $n_1$, and $n_0$ give us the counts for each of these
events, as a function of $i$, $h$, and $n$. With these components, we
can now get the number of distinguishable ways to balls to our three
\emph{classes}: $n_0$, $n_1$, and $n_2$. The number of ways of
assigning $n$ objects to classes of sizes $n_0$, $n_1$, and $n_2$ is
given by the multinomial coefficient:

\begin{equation} \label{eq:03}
\frac{n!}{n_2! n_1! n_0!}
\end{equation}

However, note that this is off by a factor of $2^{n_1}$, because each
of the $n_1$ heterozygotes can be arranged 2 ways, not just one. Thus
we multiple by $2^{n_1}$, which now counts \emph{all} the permutations
of heterozygotes. This is quite important, as we are also counting all
possible permutations of balls in heterozygotes in the binomial
coefficient above too.

Finally, this leaves \autoref{eq:schraiber:01}, which is simply a
rearranged \autoref{eq:02} over \autoref{eq:01}.

\begin{equation} \label{eq:schraiber:01}
  \text{P}(h \mid i, n) = \frac{i!(2n-1)!}{(2n)!}\frac{2^h n!}{((i-h)/2)!h!((2n - i - h)/2)!}
\end{equation}

\section{Single Inbreeding Coefficient for a Sample}

Next, Schraiber looks at a single inbreed coefficient for the sample,
$F$. He uses $F$ as the probability of being identical by descent and
$1-F$ as the complement of this. A probabilistic interpretation of $F$
requires $0 \le F \le 1$ (no excess of heterozygotes).

Schraiber assumes that the $j$ individuals at a polymorphic site that
are i.b.d is binomially distributed with $p = F$. Thus:

\begin{equation} \label{eq:schraiber:02}
\text{P}(j \mid n', F) = {n' \choose j} F^j (1-F)^{n' - j}
\end{equation}

Which is a standard binomial, with one minor caveat. Notice that we
use $n'$ here instead of $n$. This is to handle a constraint: if $i$
(the number of derived alleles) is odd, then there has to be at least
one heterozygous site. We define $n' = n$ if $i$ is even, $n = n - 1$
if $i$ is odd.

Now, let $k$ be the number of individuals that are homozygous for the
derived allele because they are i.b.d (given we know $j$). $k$ has a
hypergeometric distribution (note that there was another typo in the
original paper here):

\begin{equation} \label{eq:schraiber:03}
\text{P}(k \mid j, i', n') = \frac{  {i'/2 \choose k} { n' - i' \choose j - k } } { {n' \choose j}}
\end{equation}

This is saying, in the context of a hypergeometric distribution, if I
draw $j$ individuals (where $j$ is the number of i.b.d. individuals,
either derived or ancestral alleles), what's the probability of seeing
$k$ autozygous derived alleles given there are $i/2$ pairs to choose
from in a population of size $n$ (note that the model assumes some
model of reproduction such that autozygosity can't occur by drawing
the same derived allele twice, hence we divide by two)? To me, it
seemed strange that we worry about draws of size $j$, but our
reasoning for this is is that this entire equation will become used as
a conditional in a joint density, and summed across all possible $j$
since we are uncertain about it.

Given $j$ (the number of i.b.d. individuals) and $k$ (the number of
derived i.b.d. alleles), the number of individuals heterozygous at a
site has the distribution given by \autoref{eq:schraiber:01} with $n =
n - j$ and $i = i - 2k$ (the paper says $j = j - 2k$ but there's no
$j$ in equation (1), so I believe this is another typo). 

Finally, we want an expression for $P(h \mid i, n, F)$ (and we see
$h$, $i$, and $n$ in our real data), so we take the marginal densities
over the unknowns $j$ (number of i.b.d individuals) and $k$ (number of
autozygous derived) in our joint likelihood built up from the
conditionals:

\begin{equation} \label{eq:schraiber:04}
\text{P}(h \mid i, n, F) = \sum_{j=0}^{n'}\sum_{k=0}^{j} P(h, i - 2k, n - j) P(j \mid F, n') P(k \mid j, i', n')
\end{equation}

Thus, this serves as our likelihood model to estimate $F$ using MLE
approaches.

\subsection{$\hat{F_i}$: $F$ per a Certain Number of Derived Alleles, Across Sites}

\autoref{eq:schraiber:04} can then be used as a likelihood for each
derived allele $i$, by ``replacing $n$ by $n_i$, the observed number
of sites at which there are $i$ derived alleles, and $h$ by $h_i$, the
number of sites that are heterozygous.'' The natural way to think
about this is that we are working across sites, rather than
across individuals.

Why would this be interesting? For a sample $n$, $i$ implies a
genotype frequency ($i/2n$). For certain sites that have the same
number of derived alleles $i$ (same frequeny), and we want to estimate
$F$, as if $i$ is low (i.e. derived mutations are recent, think
neutral coalecent) a higher $F$ may indicate selection. In fact,
\citet{Schraiber:2012it} use LRT to determine if a $F$ per $i$ fits
better than a universal $F$ and the find it does. However, as the
paper shows, even allowing for $F$ for each $i$, the data are not
consistent with inbreeding coefficients being the same for each
individual. This is why they move on to estimate individual-specific
inbreeding coefficient.

\section{Individual Specific Inbreeding Coefficients}

Rather than just estimate $F_i$, one could estimate $F_j$, the
inbreeding coefficient per individual $j \in \{1, \ldots, n\}$ (note
this a different $j$ than before!). Then, we can find the set $(F_1,
\ldots, F_n)$ given $i$ and $(h_1, \ldots, h_n)$ where $h_j = (h_{j1},
h_{j2}, \ldots, h_{jL})$ in which $h_{jl}$ = 1 if individual $j$ is
heterozygous at site $l$ and 0 otherwise (note: don't we need $i_l$?).

The exact likelihood is hard, so we compute the \emph{composite 
likliehood}. Composite likelihoods are a type of pseudo-likelihood
because they don't fully specify the \emph{real} likelihood
(probabilistic model). Borrowing an example similar to one in Pawitan
2013, assume you have a linear spatial distribution of something and
you observe some array of data, $y_1, y_2, \ldots, y_n$. The joint
probability we reach for would be: $p(y_1, y_2, \ldots, y_n) = p(y_1)
p(y_2 \mid y_1) \ldots p(y_n \mid y_{n-1}, \ldots)$. However, this
imposes a right to left model of our spatial distribution, and seems
awkward. It would be more natural to think of first-order neighbor
model $p(y_k \mid y_1, \ldots, y_{k-1}, y_{k+1}, \ldots, y_n) = p(y_k
\mid y_{k-1}, y_{k+1})$. But, the product of all the conditional
probabilities is not the same as the likelihood, but instead a
composite likelihood. Pawitan gives a nice example for the simple case
where observe $(y_1, y_2)$:

$$ p(y_1 \mid y_2) p(y_2 \mid y_1)$$

This is not a likelihood; the true likelihood is:

$$ p(y_1) p(y_2 \mid y_1) $$

The former case is the composite liklihood, as is similar to the one
used in Schraiber et al.:

\begin{equation} \label{eq:schraiber:06}
\text{P}(h_1, h_2, \ldots, h_n \mid F_1, F_2, \ldots, F_n, i) = \\
\prod_{j=1}^n \prod_{l=1}^L \text{P}(h_{jl} = 1 \mid F_j, i_l)^{h_{jl}} \left( 1 - \text{P}(h_{jl} = 1 \mid F_j, i_l)\right)^{1 - h_{jl}}
\end{equation}

where $L$ is the number of polymorphic sites (note I added an $l$
subscript, as this is at a particular site and I think this is a typo
in the original). Also:

\begin{equation} \label{eq:schraiber:07}
\text{P}(h_{jl} = 1 \mid F_j, i) = (1 - F_j) \frac{i (2n - i)} {n (2n - 1)}
\end{equation}

Note that for some individual $j$'s $F_j$, this probability of being
heterozygous has the same $1 - F$ factor as the frequency of a
heterozygote under Generalized Hardy-Weinberg ($(1-F)2p(1 - p)$). So
we might naturally ask, what is $(i(2n - i))/(n(2n - 1))$? Well, the
frequency of derived alleles is $i/(2n - 1)$ and the frequency of
ancestral is $(2n - i)/2n$. I believe the frequency of $i$ has the
denominator $2n - 1$ because the condition mentioned earlier.


\bibliographystyle{plainnat}
\bibliography{schraiber-et-al-2012-bib}

\end{document}
