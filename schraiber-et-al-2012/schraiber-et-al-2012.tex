\documentclass[12pt]{article}\usepackage[]{graphicx}\usepackage[]{color}
%% maxwidth is the original width if it is less than linewidth
%% otherwise use linewidth (to make sure the graphics do not exceed the margin)
\makeatletter
\def\maxwidth{ %
  \ifdim\Gin@nat@width>\linewidth
    \linewidth
  \else
    \Gin@nat@width
  \fi
}
\makeatother

\definecolor{fgcolor}{rgb}{0.345, 0.345, 0.345}
\newcommand{\hlnum}[1]{\textcolor[rgb]{0.686,0.059,0.569}{#1}}%
\newcommand{\hlstr}[1]{\textcolor[rgb]{0.192,0.494,0.8}{#1}}%
\newcommand{\hlcom}[1]{\textcolor[rgb]{0.678,0.584,0.686}{\textit{#1}}}%
\newcommand{\hlopt}[1]{\textcolor[rgb]{0,0,0}{#1}}%
\newcommand{\hlstd}[1]{\textcolor[rgb]{0.345,0.345,0.345}{#1}}%
\newcommand{\hlkwa}[1]{\textcolor[rgb]{0.161,0.373,0.58}{\textbf{#1}}}%
\newcommand{\hlkwb}[1]{\textcolor[rgb]{0.69,0.353,0.396}{#1}}%
\newcommand{\hlkwc}[1]{\textcolor[rgb]{0.333,0.667,0.333}{#1}}%
\newcommand{\hlkwd}[1]{\textcolor[rgb]{0.737,0.353,0.396}{\textbf{#1}}}%

\usepackage{framed}
\makeatletter
\newenvironment{kframe}{%
 \def\at@end@of@kframe{}%
 \ifinner\ifhmode%
  \def\at@end@of@kframe{\end{minipage}}%
  \begin{minipage}{\columnwidth}%
 \fi\fi%
 \def\FrameCommand##1{\hskip\@totalleftmargin \hskip-\fboxsep
 \colorbox{shadecolor}{##1}\hskip-\fboxsep
     % There is no \\@totalrightmargin, so:
     \hskip-\linewidth \hskip-\@totalleftmargin \hskip\columnwidth}%
 \MakeFramed {\advance\hsize-\width
   \@totalleftmargin\z@ \linewidth\hsize
   \@setminipage}}%
 {\par\unskip\endMakeFramed%
 \at@end@of@kframe}
\makeatother

\definecolor{shadecolor}{rgb}{.97, .97, .97}
\definecolor{messagecolor}{rgb}{0, 0, 0}
\definecolor{warningcolor}{rgb}{1, 0, 1}
\definecolor{errorcolor}{rgb}{1, 0, 0}
\newenvironment{knitrout}{}{} % an empty environment to be redefined in TeX

\usepackage{alltt}
\usepackage[numbers]{natbib}
\usepackage{graphicx}
\usepackage{hyperref}
\usepackage{fullpage}
\usepackage{amsmath}
\IfFileExists{upquote.sty}{\usepackage{upquote}}{}

\begin{document}

\title{Notes on Schraiber et al., 2012 - \emph{Genomic Tests on Variation in Inbreeding Among Individuals and Among Chromosomes}}
\author{Vince Buffalo}
\maketitle

\section{Levene's 1949 Model: The Probability of Heterzygotes with No Inbreeding}

First, we consider Schraiber's \autoref{eq:schraiber:01}, the
probability of heterzygotes with no inbreeding derived from
\citep{Levene:1949va}. The setup is that we have $n$ diploid
individuals, giving us $2n$ possible ``bins'' to put $2n$ balls
into. If we only consider polymorphic sites (assuming biallelic),
there are two alleles: ancestral and derived. For our $2n$ alleles in
our sample, call $i$ the number of derived alleles. $i$ is bound
between $1$ and $2n-1$ (that is, $i \in \{1, 2, \ldots, 2n-1\}$) since
if $i=0$, we have no derived alleles and our site is not polymorphic,
and if $i=2n$, our site is entirely derived alleles, and again, our
site is no polymorphic.

There are three additional constraints. First, if we define $h$ as the
number of heterozygous individuals (that is, individuals with one of
the $i$ derived alleles and one of the $2n-i$ ancestral alleles), then
if $h$ is odd, $i$ must be odd, and second, if $h$ is even, then $i$
must be even. If this is unclear, consider the case where $h$ is
even. If an additional derived allele is added, it either goes into a
heterozygous individual, and makes it homozygous (so $h' = h - 1$,
making $h$ odd) or it makes a homozygous individual heterzygous (so
$h' = h + 1$, making $h$ odd). Finally, the last constraint is that $h
\le \min(i, 2n-i)$, which can be understood as the fewest number of
heterzygotes you can have is the minimum number of derived or
ancestral alleles. This is easy intuitively too: in the extreme case
that if \emph{every} minor allele ($i$ or $2n-i$) was in a
heterozgyous individual, then the number of minor alleles is the most
possible heterozgotes you could have.

Schraiber uses a classic colored ball and bin combinatorial approach
equation borrowed from \citep{Haldane:1954ts} to look at the
probability of heterozygotes given $i$ and $n$.


First, disregard the fact that there
are bins (our diploid individuals) and note that there are $(2n)!$
ways of rearranging all the colored balls.

 model to derive his
equation \autoref{eq:schraiber:01}. 

\begin{equation} \label{eq:schraiber:01}
  \text{Pr}(h \mid i, n) = \frac{i!(2n-1)!}{(2n)!}
\end{equation}

\bibliographystyle{plainnat}
\bibliography{schraiber-et-al-2012-bib}

\end{document}
